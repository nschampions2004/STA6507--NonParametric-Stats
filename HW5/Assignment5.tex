\documentclass[]{tufte-handout}

% ams
\usepackage{amssymb,amsmath}

\usepackage{ifxetex,ifluatex}
\usepackage{fixltx2e} % provides \textsubscript
\ifnum 0\ifxetex 1\fi\ifluatex 1\fi=0 % if pdftex
  \usepackage[T1]{fontenc}
  \usepackage[utf8]{inputenc}
\else % if luatex or xelatex
  \makeatletter
  \@ifpackageloaded{fontspec}{}{\usepackage{fontspec}}
  \makeatother
  \defaultfontfeatures{Ligatures=TeX,Scale=MatchLowercase}
  \makeatletter
  \@ifpackageloaded{soul}{
     \renewcommand\allcapsspacing[1]{{\addfontfeature{LetterSpace=15}#1}}
     \renewcommand\smallcapsspacing[1]{{\addfontfeature{LetterSpace=10}#1}}
   }{}
  \makeatother

\fi

% graphix
\usepackage{graphicx}
\setkeys{Gin}{width=\linewidth,totalheight=\textheight,keepaspectratio}

% booktabs
\usepackage{booktabs}

% url
\usepackage{url}

% hyperref
\usepackage{hyperref}

% units.
\usepackage{units}


\setcounter{secnumdepth}{-1}

% citations
\usepackage{natbib}
\bibliographystyle{plainnat}

% pandoc syntax highlighting
\usepackage{color}
\usepackage{fancyvrb}
\newcommand{\VerbBar}{|}
\newcommand{\VERB}{\Verb[commandchars=\\\{\}]}
\DefineVerbatimEnvironment{Highlighting}{Verbatim}{commandchars=\\\{\}}
% Add ',fontsize=\small' for more characters per line
\newenvironment{Shaded}{}{}
\newcommand{\KeywordTok}[1]{\textcolor[rgb]{0.00,0.44,0.13}{\textbf{#1}}}
\newcommand{\DataTypeTok}[1]{\textcolor[rgb]{0.56,0.13,0.00}{#1}}
\newcommand{\DecValTok}[1]{\textcolor[rgb]{0.25,0.63,0.44}{#1}}
\newcommand{\BaseNTok}[1]{\textcolor[rgb]{0.25,0.63,0.44}{#1}}
\newcommand{\FloatTok}[1]{\textcolor[rgb]{0.25,0.63,0.44}{#1}}
\newcommand{\ConstantTok}[1]{\textcolor[rgb]{0.53,0.00,0.00}{#1}}
\newcommand{\CharTok}[1]{\textcolor[rgb]{0.25,0.44,0.63}{#1}}
\newcommand{\SpecialCharTok}[1]{\textcolor[rgb]{0.25,0.44,0.63}{#1}}
\newcommand{\StringTok}[1]{\textcolor[rgb]{0.25,0.44,0.63}{#1}}
\newcommand{\VerbatimStringTok}[1]{\textcolor[rgb]{0.25,0.44,0.63}{#1}}
\newcommand{\SpecialStringTok}[1]{\textcolor[rgb]{0.73,0.40,0.53}{#1}}
\newcommand{\ImportTok}[1]{#1}
\newcommand{\CommentTok}[1]{\textcolor[rgb]{0.38,0.63,0.69}{\textit{#1}}}
\newcommand{\DocumentationTok}[1]{\textcolor[rgb]{0.73,0.13,0.13}{\textit{#1}}}
\newcommand{\AnnotationTok}[1]{\textcolor[rgb]{0.38,0.63,0.69}{\textbf{\textit{#1}}}}
\newcommand{\CommentVarTok}[1]{\textcolor[rgb]{0.38,0.63,0.69}{\textbf{\textit{#1}}}}
\newcommand{\OtherTok}[1]{\textcolor[rgb]{0.00,0.44,0.13}{#1}}
\newcommand{\FunctionTok}[1]{\textcolor[rgb]{0.02,0.16,0.49}{#1}}
\newcommand{\VariableTok}[1]{\textcolor[rgb]{0.10,0.09,0.49}{#1}}
\newcommand{\ControlFlowTok}[1]{\textcolor[rgb]{0.00,0.44,0.13}{\textbf{#1}}}
\newcommand{\OperatorTok}[1]{\textcolor[rgb]{0.40,0.40,0.40}{#1}}
\newcommand{\BuiltInTok}[1]{#1}
\newcommand{\ExtensionTok}[1]{#1}
\newcommand{\PreprocessorTok}[1]{\textcolor[rgb]{0.74,0.48,0.00}{#1}}
\newcommand{\AttributeTok}[1]{\textcolor[rgb]{0.49,0.56,0.16}{#1}}
\newcommand{\RegionMarkerTok}[1]{#1}
\newcommand{\InformationTok}[1]{\textcolor[rgb]{0.38,0.63,0.69}{\textbf{\textit{#1}}}}
\newcommand{\WarningTok}[1]{\textcolor[rgb]{0.38,0.63,0.69}{\textbf{\textit{#1}}}}
\newcommand{\AlertTok}[1]{\textcolor[rgb]{1.00,0.00,0.00}{\textbf{#1}}}
\newcommand{\ErrorTok}[1]{\textcolor[rgb]{1.00,0.00,0.00}{\textbf{#1}}}
\newcommand{\NormalTok}[1]{#1}

% longtable

% multiplecol
\usepackage{multicol}

% strikeout
\usepackage[normalem]{ulem}

% morefloats
\usepackage{morefloats}


% tightlist macro required by pandoc >= 1.14
\providecommand{\tightlist}{%
  \setlength{\itemsep}{0pt}\setlength{\parskip}{0pt}}

% title / author / date
\title{Assignment 5}
\author{Kyle Ligon}
\date{2018-11-19}


\begin{document}

\maketitle




\section{Problem A- Chi Square Test for Differences in
Probabilities}\label{problem-a--chi-square-test-for-differences-in-probabilities}

\begin{Shaded}
\begin{Highlighting}[]
\NormalTok{men <-}\StringTok{ }\KeywordTok{c}\NormalTok{(}\DecValTok{32}\NormalTok{, }\DecValTok{68}\NormalTok{)}
\NormalTok{women <-}\StringTok{ }\KeywordTok{c}\NormalTok{(}\DecValTok{26}\NormalTok{, }\DecValTok{74}\NormalTok{)}

\NormalTok{taste <-}\StringTok{ }\KeywordTok{data.frame}\NormalTok{(men, women, }\DataTypeTok{row.names =} \KeywordTok{c}\NormalTok{(}\StringTok{"no likey"}\NormalTok{, }
    \StringTok{"likey"}\NormalTok{))}
\NormalTok{taste_test <-}\StringTok{ }\KeywordTok{chisq.test}\NormalTok{(}\DataTypeTok{x =}\NormalTok{ taste, }\DataTypeTok{correct =} \OtherTok{FALSE}\NormalTok{) }\OperatorTok\StringTok{ }
\StringTok{    }\KeywordTok{tidy}\NormalTok{()}
\end{Highlighting}
\end{Shaded}

\subsection{Hypotheses:}\label{hypotheses}

\(H_{0}\): \(p_{men}\) = \(p_{women}\)

\(H_{1}\): \(p_{men}\) \neq \(p_{women}\)

\subsection{Test Statistic}\label{test-statistic}

The test statistic of this Chi-Square test is 1

\subsection{Critical Region}\label{critical-region}

We are looking for a \(\chi_{0.975, 1, 1}^2\) and
\(\chi_{0.025, 1, 1}^2\), which is equal to 8.7652.

\subsection{Conclusion}\label{conclusion}

With the test statistic not greater than or less than the critical
region, we cannot reject the Null hypothesis that the probabilities are
the same. There is not enough evidence to suggest that the two
probabilities are different.

\section{Problem B- Fisher's Exact
Test}\label{problem-b--fishers-exact-test}

\subsection{Hypotheses:}\label{hypotheses-1}

\(H_{0}\): All \emph{c} populations have the same median.

\(H_{1}\): At least two of the populations have different medians.

\subsection{Test Statistic}\label{test-statistic-1}

The test statistic of this Median test is 0.8269.

\subsection{Critical Region}\label{critical-region-1}

We are looking for a \(\chi_{0.95, 2}^2\), which is equal to 5.9941.

\subsection{Conclusion}\label{conclusion-1}

With the test statistic greater than the critical region, we can reject
the Null hypothesis that the medians are the same. There is evidence to
suggest that at least two medians are different.

\section{Problem C- Chi Square Test for Differences in
Probabilities}\label{problem-c--chi-square-test-for-differences-in-probabilities}

\begin{Shaded}
\begin{Highlighting}[]
\NormalTok{ase <-}\StringTok{ }\KeywordTok{c}\NormalTok{(}\DecValTok{11}\NormalTok{, }\DecValTok{11}\NormalTok{, }\DecValTok{1}\NormalTok{)}
\NormalTok{nyse <-}\StringTok{ }\KeywordTok{c}\NormalTok{(}\DecValTok{24}\NormalTok{, }\DecValTok{11}\NormalTok{, }\DecValTok{0}\NormalTok{)}

\NormalTok{stocks <-}\StringTok{ }\KeywordTok{data.frame}\NormalTok{(ase, nyse, }\DataTypeTok{row.names =} \KeywordTok{c}\NormalTok{(}\StringTok{"A"}\NormalTok{, }
    \StringTok{"B"}\NormalTok{, }\StringTok{"C"}\NormalTok{)) }\OperatorTok\StringTok{ }\KeywordTok{t}\NormalTok{()}

\NormalTok{stock_test <-}\StringTok{ }\KeywordTok{chisq.test}\NormalTok{(}\DataTypeTok{x =}\NormalTok{ stocks, }\DataTypeTok{correct =} \OtherTok{FALSE}\NormalTok{)}
\end{Highlighting}
\end{Shaded}

\subsection{Hypotheses:}\label{hypotheses-2}

\(H_{0}\): \(p_{ASE}\) = \(p_{NYSE}\)

\(H_{1}\): At least two of the populations have different populations.

\subsection{Test Statistic}\label{test-statistic-2}

The test statistic of this Median test is 0.8269.

\subsection{Critical Region}\label{critical-region-2}

We are looking for a \(\chi_{0.95, 2}^2\), which is equal to 5.9941.

\subsection{Conclusion}\label{conclusion-2}

With the test statistic greater than the critical region, we can reject
the Null hypothesis that the medians are the same. There is evidence to
suggest that at least two medians are different.

\section{Problem D- Chi-Square Test}\label{problem-d--chi-square-test}

\subsection{Hypotheses:}\label{hypotheses-3}

\(H_{0}\): All \emph{c} populations have the same median.

\(H_{1}\): At least two of the populations have different medians.

\subsection{Test Statistic}\label{test-statistic-3}

The test statistic of this Median test is 0.8269.

\subsection{Critical Region}\label{critical-region-3}

We are looking for a \(\chi_{0.95, 2}^2\), which is equal to 5.9941.

\subsection{Conclusion}\label{conclusion-3}

With the test statistic greater than the critical region, we can reject
the Null hypothesis that the medians are the same. There is evidence to
suggest that at least two medians are different.

\section{Problem E- Median Test}\label{problem-e--median-test}

\begin{Shaded}
\begin{Highlighting}[]
\CommentTok{# packages needed}
\KeywordTok{library}\NormalTok{(agricolae)}

\NormalTok{sampl_}\DecValTok{1}\NormalTok{ <-}\StringTok{ }\KeywordTok{c}\NormalTok{(}\DecValTok{35}\NormalTok{, }\DecValTok{42}\NormalTok{, }\DecValTok{42}\NormalTok{, }\DecValTok{30}\NormalTok{, }\DecValTok{15}\NormalTok{, }\DecValTok{31}\NormalTok{, }\DecValTok{29}\NormalTok{, }\DecValTok{29}\NormalTok{, }\DecValTok{17}\NormalTok{)}
\NormalTok{sampl_}\DecValTok{2}\NormalTok{ <-}\StringTok{ }\KeywordTok{c}\NormalTok{(}\DecValTok{34}\NormalTok{, }\DecValTok{38}\NormalTok{, }\DecValTok{26}\NormalTok{, }\DecValTok{17}\NormalTok{, }\DecValTok{42}\NormalTok{, }\DecValTok{28}\NormalTok{, }\DecValTok{35}\NormalTok{, }\DecValTok{33}\NormalTok{, }\DecValTok{16}\NormalTok{)}
\NormalTok{sampl_}\DecValTok{3}\NormalTok{ <-}\StringTok{ }\KeywordTok{c}\NormalTok{(}\DecValTok{17}\NormalTok{, }\DecValTok{29}\NormalTok{, }\DecValTok{30}\NormalTok{, }\DecValTok{36}\NormalTok{, }\DecValTok{41}\NormalTok{, }\DecValTok{30}\NormalTok{, }\DecValTok{31}\NormalTok{, }\DecValTok{23}\NormalTok{, }\DecValTok{38}\NormalTok{)}

\NormalTok{sample_col <-}\StringTok{ }\KeywordTok{c}\NormalTok{(}\KeywordTok{rep}\NormalTok{(}\StringTok{"samp_1"}\NormalTok{, }\KeywordTok{length}\NormalTok{(sampl_}\DecValTok{1}\NormalTok{)), }
    \KeywordTok{rep}\NormalTok{(}\StringTok{"samp_2"}\NormalTok{, }\KeywordTok{length}\NormalTok{(sampl_}\DecValTok{2}\NormalTok{)), }\KeywordTok{rep}\NormalTok{(}\StringTok{"samp_3"}\NormalTok{, }
        \KeywordTok{length}\NormalTok{(sampl_}\DecValTok{3}\NormalTok{)))}

\NormalTok{medians <-}\StringTok{ }\KeywordTok{c}\NormalTok{(sampl_}\DecValTok{1}\NormalTok{, sampl_}\DecValTok{2}\NormalTok{, sampl_}\DecValTok{3}\NormalTok{)}

\NormalTok{med_frame <-}\StringTok{ }\KeywordTok{data.frame}\NormalTok{(}\DataTypeTok{sample_col =} \KeywordTok{as.factor}\NormalTok{(sample_col), }
    \DataTypeTok{medians =}\NormalTok{ medians) }\OperatorTok\StringTok{ }\KeywordTok{as.tibble}\NormalTok{()}

\NormalTok{med_test <-}\StringTok{ }\KeywordTok{Median.test}\NormalTok{(}\DataTypeTok{trt =}\NormalTok{ med_frame}\OperatorTok{$}\NormalTok{sample_col, }
    \DataTypeTok{y =}\NormalTok{ med_frame}\OperatorTok{$}\NormalTok{medians)}
\end{Highlighting}
\end{Shaded}

\subsection{Hypotheses:}\label{hypotheses-4}

\(H_{0}\): All \emph{c} populations have the same median.

\(H_{1}\): At least two of the populations have different medians.

\subsection{Test Statistic}\label{test-statistic-4}

The test statistic of this Median test is 0.8269.

\subsection{Critical Region}\label{critical-region-4}

We are looking for a \(\chi_{0.95, 2}^2\), which is equal to 5.9941.

\subsection{Conclusion}\label{conclusion-4}

With the test statistic greater than the critical region, we can reject
the Null hypothesis that the medians are the same. There is evidence to
suggest that at least two medians are different.

\bibliography{skeleton.bib}



\end{document}
