\documentclass[]{article}
\usepackage{lmodern}
\usepackage{amssymb,amsmath}
\usepackage{ifxetex,ifluatex}
\usepackage{fixltx2e} % provides \textsubscript
\ifnum 0\ifxetex 1\fi\ifluatex 1\fi=0 % if pdftex
  \usepackage[T1]{fontenc}
  \usepackage[utf8]{inputenc}
\else % if luatex or xelatex
  \ifxetex
    \usepackage{mathspec}
  \else
    \usepackage{fontspec}
  \fi
  \defaultfontfeatures{Ligatures=TeX,Scale=MatchLowercase}
\fi
% use upquote if available, for straight quotes in verbatim environments
\IfFileExists{upquote.sty}{\usepackage{upquote}}{}
% use microtype if available
\IfFileExists{microtype.sty}{%
\usepackage{microtype}
\UseMicrotypeSet[protrusion]{basicmath} % disable protrusion for tt fonts
}{}
\usepackage[margin=1in]{geometry}
\usepackage{hyperref}
\hypersetup{unicode=true,
            pdftitle={Assignment 2 Take 2},
            pdfauthor={Kyle Ligon},
            pdfborder={0 0 0},
            breaklinks=true}
\urlstyle{same}  % don't use monospace font for urls
\usepackage{color}
\usepackage{fancyvrb}
\newcommand{\VerbBar}{|}
\newcommand{\VERB}{\Verb[commandchars=\\\{\}]}
\DefineVerbatimEnvironment{Highlighting}{Verbatim}{commandchars=\\\{\}}
% Add ',fontsize=\small' for more characters per line
\usepackage{framed}
\definecolor{shadecolor}{RGB}{248,248,248}
\newenvironment{Shaded}{\begin{snugshade}}{\end{snugshade}}
\newcommand{\KeywordTok}[1]{\textcolor[rgb]{0.13,0.29,0.53}{\textbf{#1}}}
\newcommand{\DataTypeTok}[1]{\textcolor[rgb]{0.13,0.29,0.53}{#1}}
\newcommand{\DecValTok}[1]{\textcolor[rgb]{0.00,0.00,0.81}{#1}}
\newcommand{\BaseNTok}[1]{\textcolor[rgb]{0.00,0.00,0.81}{#1}}
\newcommand{\FloatTok}[1]{\textcolor[rgb]{0.00,0.00,0.81}{#1}}
\newcommand{\ConstantTok}[1]{\textcolor[rgb]{0.00,0.00,0.00}{#1}}
\newcommand{\CharTok}[1]{\textcolor[rgb]{0.31,0.60,0.02}{#1}}
\newcommand{\SpecialCharTok}[1]{\textcolor[rgb]{0.00,0.00,0.00}{#1}}
\newcommand{\StringTok}[1]{\textcolor[rgb]{0.31,0.60,0.02}{#1}}
\newcommand{\VerbatimStringTok}[1]{\textcolor[rgb]{0.31,0.60,0.02}{#1}}
\newcommand{\SpecialStringTok}[1]{\textcolor[rgb]{0.31,0.60,0.02}{#1}}
\newcommand{\ImportTok}[1]{#1}
\newcommand{\CommentTok}[1]{\textcolor[rgb]{0.56,0.35,0.01}{\textit{#1}}}
\newcommand{\DocumentationTok}[1]{\textcolor[rgb]{0.56,0.35,0.01}{\textbf{\textit{#1}}}}
\newcommand{\AnnotationTok}[1]{\textcolor[rgb]{0.56,0.35,0.01}{\textbf{\textit{#1}}}}
\newcommand{\CommentVarTok}[1]{\textcolor[rgb]{0.56,0.35,0.01}{\textbf{\textit{#1}}}}
\newcommand{\OtherTok}[1]{\textcolor[rgb]{0.56,0.35,0.01}{#1}}
\newcommand{\FunctionTok}[1]{\textcolor[rgb]{0.00,0.00,0.00}{#1}}
\newcommand{\VariableTok}[1]{\textcolor[rgb]{0.00,0.00,0.00}{#1}}
\newcommand{\ControlFlowTok}[1]{\textcolor[rgb]{0.13,0.29,0.53}{\textbf{#1}}}
\newcommand{\OperatorTok}[1]{\textcolor[rgb]{0.81,0.36,0.00}{\textbf{#1}}}
\newcommand{\BuiltInTok}[1]{#1}
\newcommand{\ExtensionTok}[1]{#1}
\newcommand{\PreprocessorTok}[1]{\textcolor[rgb]{0.56,0.35,0.01}{\textit{#1}}}
\newcommand{\AttributeTok}[1]{\textcolor[rgb]{0.77,0.63,0.00}{#1}}
\newcommand{\RegionMarkerTok}[1]{#1}
\newcommand{\InformationTok}[1]{\textcolor[rgb]{0.56,0.35,0.01}{\textbf{\textit{#1}}}}
\newcommand{\WarningTok}[1]{\textcolor[rgb]{0.56,0.35,0.01}{\textbf{\textit{#1}}}}
\newcommand{\AlertTok}[1]{\textcolor[rgb]{0.94,0.16,0.16}{#1}}
\newcommand{\ErrorTok}[1]{\textcolor[rgb]{0.64,0.00,0.00}{\textbf{#1}}}
\newcommand{\NormalTok}[1]{#1}
\usepackage{graphicx,grffile}
\makeatletter
\def\maxwidth{\ifdim\Gin@nat@width>\linewidth\linewidth\else\Gin@nat@width\fi}
\def\maxheight{\ifdim\Gin@nat@height>\textheight\textheight\else\Gin@nat@height\fi}
\makeatother
% Scale images if necessary, so that they will not overflow the page
% margins by default, and it is still possible to overwrite the defaults
% using explicit options in \includegraphics[width, height, ...]{}
\setkeys{Gin}{width=\maxwidth,height=\maxheight,keepaspectratio}
\IfFileExists{parskip.sty}{%
\usepackage{parskip}
}{% else
\setlength{\parindent}{0pt}
\setlength{\parskip}{6pt plus 2pt minus 1pt}
}
\setlength{\emergencystretch}{3em}  % prevent overfull lines
\providecommand{\tightlist}{%
  \setlength{\itemsep}{0pt}\setlength{\parskip}{0pt}}
\setcounter{secnumdepth}{0}
% Redefines (sub)paragraphs to behave more like sections
\ifx\paragraph\undefined\else
\let\oldparagraph\paragraph
\renewcommand{\paragraph}[1]{\oldparagraph{#1}\mbox{}}
\fi
\ifx\subparagraph\undefined\else
\let\oldsubparagraph\subparagraph
\renewcommand{\subparagraph}[1]{\oldsubparagraph{#1}\mbox{}}
\fi

%%% Use protect on footnotes to avoid problems with footnotes in titles
\let\rmarkdownfootnote\footnote%
\def\footnote{\protect\rmarkdownfootnote}

%%% Change title format to be more compact
\usepackage{titling}

% Create subtitle command for use in maketitle
\newcommand{\subtitle}[1]{
  \posttitle{
    \begin{center}\large#1\end{center}
    }
}

\setlength{\droptitle}{-2em}

  \title{Assignment 2 Take 2}
    \pretitle{\vspace{\droptitle}\centering\huge}
  \posttitle{\par}
    \author{Kyle Ligon}
    \preauthor{\centering\large\emph}
  \postauthor{\par}
      \predate{\centering\large\emph}
  \postdate{\par}
    \date{September 24, 2018}


\begin{document}
\maketitle

\begin{enumerate}
\def\labelenumi{\Alph{enumi})}
\item
\end{enumerate}

\begin{enumerate}
\def\labelenumi{\arabic{enumi}.}
\tightlist
\item
  Explain how do the bootstrap methods work?\\
  Bootstrap methods take multiple runs with replacement of a sample.
  With the created distribution, you can map out confidence intervals of
  the statistic in question.

  \begin{enumerate}
  \def\labelenumii{\arabic{enumii}.}
  \setcounter{enumii}{1}
  \tightlist
  \item
    and 3.
  \end{enumerate}
\end{enumerate}

\begin{Shaded}
\begin{Highlighting}[]
\CommentTok{# Package(s) needed }
\KeywordTok{library}\NormalTok{(boot)}
\KeywordTok{library}\NormalTok{(tidyverse)}
\KeywordTok{library}\NormalTok{(knitr)}
\NormalTok{ttSample <-}\StringTok{ }\KeywordTok{c}\NormalTok{(}\DecValTok{0}\NormalTok{,}\DecValTok{0}\NormalTok{,}\DecValTok{1}\NormalTok{,}\DecValTok{1}\NormalTok{,}\DecValTok{1}\NormalTok{,}\DecValTok{1}\NormalTok{,}\DecValTok{1}\NormalTok{,}\DecValTok{1}\NormalTok{,}\DecValTok{1}\NormalTok{,}\DecValTok{1}\NormalTok{,}\DecValTok{1}\NormalTok{,}\DecValTok{1}\NormalTok{,}\DecValTok{1}\NormalTok{,}\DecValTok{1}\NormalTok{,}\DecValTok{1}\NormalTok{,}\DecValTok{1}\NormalTok{,}\DecValTok{1}\NormalTok{,}\DecValTok{1}\NormalTok{,}\DecValTok{1}\NormalTok{,}\DecValTok{1}\NormalTok{)}

\CommentTok{# Create a function f in R to compute the observed statistic}
\NormalTok{f <-}\StringTok{ }\ControlFlowTok{function}\NormalTok{(DATA, i)\{}
  \KeywordTok{return}\NormalTok{(}\KeywordTok{mean}\NormalTok{(DATA[i]))}
\NormalTok{\}}

\CommentTok{# bootstrapping with R replications }
\NormalTok{results <-}\StringTok{ }\KeywordTok{boot}\NormalTok{(ttSample, f , }\DataTypeTok{R =} \DecValTok{250}\NormalTok{)}

\CommentTok{# get confidence interval }
\KeywordTok{boot.ci}\NormalTok{(results, }\DataTypeTok{conf =} \FloatTok{0.90}\NormalTok{, }\DataTypeTok{type=}\StringTok{"all"}\NormalTok{)}
\end{Highlighting}
\end{Shaded}

\begin{verbatim}
## BOOTSTRAP CONFIDENCE INTERVAL CALCULATIONS
## Based on 250 bootstrap replicates
## 
## CALL : 
## boot.ci(boot.out = results, conf = 0.9, type = "all")
## 
## Intervals : 
## Level      Normal              Basic         
## 90%   ( 0.7824,  1.0008 )   ( 0.8000,  1.0000 )  
## 
## Level     Percentile            BCa          
## 90%   ( 0.80,  1.00 )   ( 0.70,  0.95 )  
## Calculations and Intervals on Original Scale
## Warning : BCa Intervals used Extreme Quantiles
## Some BCa intervals may be unstable
\end{verbatim}

\begin{Shaded}
\begin{Highlighting}[]
\KeywordTok{boot.ci}\NormalTok{(results, }\DataTypeTok{conf =} \FloatTok{0.95}\NormalTok{, }\DataTypeTok{type=}\StringTok{"all"}\NormalTok{)}
\end{Highlighting}
\end{Shaded}

\begin{verbatim}
## BOOTSTRAP CONFIDENCE INTERVAL CALCULATIONS
## Based on 250 bootstrap replicates
## 
## CALL : 
## boot.ci(boot.out = results, conf = 0.95, type = "all")
## 
## Intervals : 
## Level      Normal              Basic         
## 95%   ( 0.7615,  1.0217 )   ( 0.8000,  1.0500 )  
## 
## Level     Percentile            BCa          
## 95%   ( 0.75,  1.00 )   ( 0.70,  0.95 )  
## Calculations and Intervals on Original Scale
## Some basic intervals may be unstable
## Some percentile intervals may be unstable
## Warning : BCa Intervals used Extreme Quantiles
## Some BCa intervals may be unstable
\end{verbatim}

\begin{Shaded}
\begin{Highlighting}[]
\KeywordTok{boot.ci}\NormalTok{(results, }\DataTypeTok{conf =} \FloatTok{0.99}\NormalTok{, }\DataTypeTok{type=}\StringTok{"all"}\NormalTok{)}
\end{Highlighting}
\end{Shaded}

\begin{verbatim}
## BOOTSTRAP CONFIDENCE INTERVAL CALCULATIONS
## Based on 250 bootstrap replicates
## 
## CALL : 
## boot.ci(boot.out = results, conf = 0.99, type = "all")
## 
## Intervals : 
## Level      Normal              Basic         
## 99%   ( 0.7206,  1.0626 )   ( 0.8000,  1.0840 )  
## 
## Level     Percentile            BCa          
## 99%   ( 0.7160,  1.0000 )   ( 0.7000,  0.9549 )  
## Calculations and Intervals on Original Scale
## Some basic intervals may be unstable
## Some percentile intervals may be unstable
## Warning : BCa Intervals used Extreme Quantiles
## Some BCa intervals may be unstable
\end{verbatim}

\begin{Shaded}
\begin{Highlighting}[]
\NormalTok{r <-}\StringTok{ }\KeywordTok{as.data.frame}\NormalTok{(bootMeans <-}\StringTok{ }\NormalTok{results}\OperatorTok{$}\NormalTok{t)}
\CommentTok{#plotting the histogram}
\KeywordTok{ggplot}\NormalTok{(r, }\KeywordTok{aes}\NormalTok{(}\DataTypeTok{x =}\NormalTok{ bootMeans)) }\OperatorTok{+}\StringTok{ }
\StringTok{  }\KeywordTok{geom_histogram}\NormalTok{(}\DataTypeTok{binwidth =} \FloatTok{0.05}\NormalTok{, }\DataTypeTok{bins =} \KeywordTok{round}\NormalTok{(}\KeywordTok{sqrt}\NormalTok{(}\KeywordTok{nrow}\NormalTok{(r)), }\DecValTok{0}\NormalTok{)) }\OperatorTok{+}
\StringTok{  }\KeywordTok{ggtitle}\NormalTok{(}\DataTypeTok{label =} \KeywordTok{paste0}\NormalTok{(results}\OperatorTok{$}\NormalTok{R, }\StringTok{" bootstrapped samples of the Texas Tech Grad Class"}\NormalTok{))}
\end{Highlighting}
\end{Shaded}

\includegraphics{Assignment2Take2_files/figure-latex/unnamed-chunk-1-1.pdf}

\begin{Shaded}
\begin{Highlighting}[]
\NormalTok{results <-}\StringTok{ }\KeywordTok{boot}\NormalTok{(ttSample, f , }\DataTypeTok{R =} \DecValTok{1000}\NormalTok{)}
\KeywordTok{boot.ci}\NormalTok{(results, }\DataTypeTok{conf =} \FloatTok{0.90}\NormalTok{, }\DataTypeTok{type=}\StringTok{"all"}\NormalTok{)}
\end{Highlighting}
\end{Shaded}

\begin{verbatim}
## BOOTSTRAP CONFIDENCE INTERVAL CALCULATIONS
## Based on 1000 bootstrap replicates
## 
## CALL : 
## boot.ci(boot.out = results, conf = 0.9, type = "all")
## 
## Intervals : 
## Level      Normal              Basic         
## 90%   ( 0.7860,  1.0085 )   ( 0.8000,  1.0000 )  
## 
## Level     Percentile            BCa          
## 90%   ( 0.80,  1.00 )   ( 0.65,  0.95 )  
## Calculations and Intervals on Original Scale
## Warning : BCa Intervals used Extreme Quantiles
## Some BCa intervals may be unstable
\end{verbatim}

\begin{Shaded}
\begin{Highlighting}[]
\KeywordTok{boot.ci}\NormalTok{(results, }\DataTypeTok{conf =} \FloatTok{0.95}\NormalTok{, }\DataTypeTok{type=}\StringTok{"all"}\NormalTok{)}
\end{Highlighting}
\end{Shaded}

\begin{verbatim}
## BOOTSTRAP CONFIDENCE INTERVAL CALCULATIONS
## Based on 1000 bootstrap replicates
## 
## CALL : 
## boot.ci(boot.out = results, conf = 0.95, type = "all")
## 
## Intervals : 
## Level      Normal              Basic         
## 95%   ( 0.7647,  1.0298 )   ( 0.8000,  1.0500 )  
## 
## Level     Percentile            BCa          
## 95%   ( 0.75,  1.00 )   ( 0.65,  0.95 )  
## Calculations and Intervals on Original Scale
## Warning : BCa Intervals used Extreme Quantiles
## Some BCa intervals may be unstable
\end{verbatim}

\begin{Shaded}
\begin{Highlighting}[]
\KeywordTok{boot.ci}\NormalTok{(results, }\DataTypeTok{conf =} \FloatTok{0.99}\NormalTok{, }\DataTypeTok{type=}\StringTok{"all"}\NormalTok{)}
\end{Highlighting}
\end{Shaded}

\begin{verbatim}
## BOOTSTRAP CONFIDENCE INTERVAL CALCULATIONS
## Based on 1000 bootstrap replicates
## 
## CALL : 
## boot.ci(boot.out = results, conf = 0.99, type = "all")
## 
## Intervals : 
## Level      Normal              Basic         
## 99%   ( 0.7230,  1.0715 )   ( 0.8000,  1.1000 )  
## 
## Level     Percentile            BCa          
## 99%   ( 0.70,  1.00 )   ( 0.65,  1.00 )  
## Calculations and Intervals on Original Scale
## Some basic intervals may be unstable
## Some percentile intervals may be unstable
## Warning : BCa Intervals used Extreme Quantiles
## Some BCa intervals may be unstable
\end{verbatim}

\begin{Shaded}
\begin{Highlighting}[]
\NormalTok{r <-}\StringTok{ }\KeywordTok{as.data.frame}\NormalTok{(bootMeans <-}\StringTok{ }\NormalTok{results}\OperatorTok{$}\NormalTok{t)}
\KeywordTok{ggplot}\NormalTok{(r, }\KeywordTok{aes}\NormalTok{(}\DataTypeTok{x =}\NormalTok{ bootMeans)) }\OperatorTok{+}\StringTok{ }
\StringTok{  }\KeywordTok{geom_histogram}\NormalTok{(}\DataTypeTok{binwidth =} \FloatTok{0.05}\NormalTok{) }\OperatorTok{+}
\StringTok{  }\KeywordTok{ggtitle}\NormalTok{(}\DataTypeTok{label =} \KeywordTok{paste0}\NormalTok{(results}\OperatorTok{$}\NormalTok{R, }\StringTok{" bootstrapped samples of the Texas Tech Grad Class"}\NormalTok{))}
\end{Highlighting}
\end{Shaded}

\includegraphics{Assignment2Take2_files/figure-latex/unnamed-chunk-1-2.pdf}

\begin{Shaded}
\begin{Highlighting}[]
\NormalTok{results <-}\StringTok{ }\KeywordTok{boot}\NormalTok{(ttSample, f , }\DataTypeTok{R =} \DecValTok{5000}\NormalTok{)}
\KeywordTok{boot.ci}\NormalTok{(results, }\DataTypeTok{conf =} \FloatTok{0.90}\NormalTok{, }\DataTypeTok{type=}\StringTok{"all"}\NormalTok{)}
\end{Highlighting}
\end{Shaded}

\begin{verbatim}
## BOOTSTRAP CONFIDENCE INTERVAL CALCULATIONS
## Based on 5000 bootstrap replicates
## 
## CALL : 
## boot.ci(boot.out = results, conf = 0.9, type = "all")
## 
## Intervals : 
## Level      Normal              Basic         
## 90%   ( 0.7885,  1.0107 )   ( 0.8000,  1.0000 )  
## 
## Level     Percentile            BCa          
## 90%   ( 0.80,  1.00 )   ( 0.65,  0.95 )  
## Calculations and Intervals on Original Scale
## Some BCa intervals may be unstable
\end{verbatim}

\begin{Shaded}
\begin{Highlighting}[]
\KeywordTok{boot.ci}\NormalTok{(results, }\DataTypeTok{conf =} \FloatTok{0.95}\NormalTok{, }\DataTypeTok{type=}\StringTok{"all"}\NormalTok{)}
\end{Highlighting}
\end{Shaded}

\begin{verbatim}
## BOOTSTRAP CONFIDENCE INTERVAL CALCULATIONS
## Based on 5000 bootstrap replicates
## 
## CALL : 
## boot.ci(boot.out = results, conf = 0.95, type = "all")
## 
## Intervals : 
## Level      Normal              Basic         
## 95%   ( 0.7672,  1.0320 )   ( 0.8000,  1.0500 )  
## 
## Level     Percentile            BCa          
## 95%   ( 0.75,  1.00 )   ( 0.60,  0.95 )  
## Calculations and Intervals on Original Scale
## Warning : BCa Intervals used Extreme Quantiles
## Some BCa intervals may be unstable
\end{verbatim}

\begin{Shaded}
\begin{Highlighting}[]
\KeywordTok{boot.ci}\NormalTok{(results, }\DataTypeTok{conf =} \FloatTok{0.99}\NormalTok{, }\DataTypeTok{type=}\StringTok{"all"}\NormalTok{)}
\end{Highlighting}
\end{Shaded}

\begin{verbatim}
## BOOTSTRAP CONFIDENCE INTERVAL CALCULATIONS
## Based on 5000 bootstrap replicates
## 
## CALL : 
## boot.ci(boot.out = results, conf = 0.99, type = "all")
## 
## Intervals : 
## Level      Normal              Basic         
## 99%   ( 0.7256,  1.0736 )   ( 0.8000,  1.1000 )  
## 
## Level     Percentile            BCa          
## 99%   ( 0.7,  1.0 )   ( 0.6,  1.0 )  
## Calculations and Intervals on Original Scale
## Warning : BCa Intervals used Extreme Quantiles
## Some BCa intervals may be unstable
\end{verbatim}

\begin{Shaded}
\begin{Highlighting}[]
\NormalTok{r <-}\StringTok{ }\KeywordTok{as.data.frame}\NormalTok{(bootMeans <-}\StringTok{ }\NormalTok{results}\OperatorTok{$}\NormalTok{t)}
\KeywordTok{ggplot}\NormalTok{(r, }\KeywordTok{aes}\NormalTok{(}\DataTypeTok{x =}\NormalTok{ bootMeans)) }\OperatorTok{+}\StringTok{ }
\StringTok{  }\KeywordTok{geom_histogram}\NormalTok{(}\DataTypeTok{binwidth =} \FloatTok{0.05}\NormalTok{) }\OperatorTok{+}
\StringTok{  }\KeywordTok{ggtitle}\NormalTok{(}\DataTypeTok{label =} \KeywordTok{paste0}\NormalTok{(results}\OperatorTok{$}\NormalTok{R, }\StringTok{" bootstrapped samples of the Texas Tech Grad Class"}\NormalTok{))}
\end{Highlighting}
\end{Shaded}

\includegraphics{Assignment2Take2_files/figure-latex/unnamed-chunk-1-3.pdf}

\begin{Shaded}
\begin{Highlighting}[]
\NormalTok{results <-}\StringTok{ }\KeywordTok{boot}\NormalTok{(ttSample, f , }\DataTypeTok{R =} \DecValTok{10000}\NormalTok{)}
\KeywordTok{boot.ci}\NormalTok{(results, }\DataTypeTok{conf =} \FloatTok{0.90}\NormalTok{, }\DataTypeTok{type=}\StringTok{"all"}\NormalTok{)}
\end{Highlighting}
\end{Shaded}

\begin{verbatim}
## BOOTSTRAP CONFIDENCE INTERVAL CALCULATIONS
## Based on 10000 bootstrap replicates
## 
## CALL : 
## boot.ci(boot.out = results, conf = 0.9, type = "all")
## 
## Intervals : 
## Level      Normal              Basic         
## 90%   ( 0.7909,  1.0106 )   ( 0.8000,  1.0000 )  
## 
## Level     Percentile            BCa          
## 90%   ( 0.80,  1.00 )   ( 0.65,  0.95 )  
## Calculations and Intervals on Original Scale
## Some BCa intervals may be unstable
\end{verbatim}

\begin{Shaded}
\begin{Highlighting}[]
\KeywordTok{boot.ci}\NormalTok{(results, }\DataTypeTok{conf =} \FloatTok{0.95}\NormalTok{, }\DataTypeTok{type=}\StringTok{"all"}\NormalTok{)}
\end{Highlighting}
\end{Shaded}

\begin{verbatim}
## BOOTSTRAP CONFIDENCE INTERVAL CALCULATIONS
## Based on 10000 bootstrap replicates
## 
## CALL : 
## boot.ci(boot.out = results, conf = 0.95, type = "all")
## 
## Intervals : 
## Level      Normal              Basic         
## 95%   ( 0.7699,  1.0317 )   ( 0.8000,  1.0500 )  
## 
## Level     Percentile            BCa          
## 95%   ( 0.75,  1.00 )   ( 0.60,  0.95 )  
## Calculations and Intervals on Original Scale
## Some BCa intervals may be unstable
\end{verbatim}

\begin{Shaded}
\begin{Highlighting}[]
\KeywordTok{boot.ci}\NormalTok{(results, }\DataTypeTok{conf =} \FloatTok{0.99}\NormalTok{, }\DataTypeTok{type=}\StringTok{"all"}\NormalTok{)}
\end{Highlighting}
\end{Shaded}

\begin{verbatim}
## BOOTSTRAP CONFIDENCE INTERVAL CALCULATIONS
## Based on 10000 bootstrap replicates
## 
## CALL : 
## boot.ci(boot.out = results, conf = 0.99, type = "all")
## 
## Intervals : 
## Level      Normal              Basic         
## 99%   ( 0.7287,  1.0728 )   ( 0.8000,  1.1000 )  
## 
## Level     Percentile            BCa          
## 99%   ( 0.7,  1.0 )   ( 0.6,  1.0 )  
## Calculations and Intervals on Original Scale
## Warning : BCa Intervals used Extreme Quantiles
## Some BCa intervals may be unstable
\end{verbatim}

\begin{Shaded}
\begin{Highlighting}[]
\NormalTok{r <-}\StringTok{ }\KeywordTok{as.data.frame}\NormalTok{(bootMeans <-}\StringTok{ }\NormalTok{results}\OperatorTok{$}\NormalTok{t)}
\KeywordTok{ggplot}\NormalTok{(r, }\KeywordTok{aes}\NormalTok{(}\DataTypeTok{x =}\NormalTok{ bootMeans)) }\OperatorTok{+}\StringTok{ }
\StringTok{  }\KeywordTok{geom_histogram}\NormalTok{(}\DataTypeTok{binwidth =} \FloatTok{0.05}\NormalTok{) }\OperatorTok{+}
\StringTok{  }\KeywordTok{ggtitle}\NormalTok{(}\DataTypeTok{label =} \KeywordTok{paste0}\NormalTok{(results}\OperatorTok{$}\NormalTok{R, }\StringTok{" bootstrapped samples of the Texas Tech Grad Class"}\NormalTok{))}
\end{Highlighting}
\end{Shaded}

\includegraphics{Assignment2Take2_files/figure-latex/unnamed-chunk-1-4.pdf}

\begin{enumerate}
\def\labelenumi{\arabic{enumi}.}
\setcounter{enumi}{3}
\tightlist
\item
  Discuss and comment on the results.\\
  As our sample size increases, we see the distributions center more
  closely around p\_hat = 0.90. Additionally, the standard errors
  tighten up as the sample size increases. finally, the confidence
  intervals shrink accordingly as the bootstrap samples increase.
\end{enumerate}

\begin{enumerate}
\def\labelenumi{\Alph{enumi})}
\setcounter{enumi}{1}
\item
  \begin{enumerate}
  \def\labelenumii{\arabic{enumii}.}
  \tightlist
  \item
    Generate n in \{5, 10, 30\} samples from the standard normal
    distribution. Then, calculate the confidence intervals from each.
  \end{enumerate}
\end{enumerate}

\begin{Shaded}
\begin{Highlighting}[]
\CommentTok{# 5 samples from standard normal distribution}
\KeywordTok{set.seed}\NormalTok{(}\DecValTok{1}\NormalTok{)}
\NormalTok{n_}\DecValTok{5}\NormalTok{ <-}\StringTok{ }\KeywordTok{rnorm}\NormalTok{(}\DecValTok{5}\NormalTok{)}

\NormalTok{confidence_int <-}\StringTok{ }\ControlFlowTok{function}\NormalTok{(vector)\{}
  \CommentTok{#high value of the confidence interval}
\NormalTok{  high_val <-}\StringTok{ }\KeywordTok{round}\NormalTok{(}\KeywordTok{mean}\NormalTok{(vector) }\OperatorTok{+}\StringTok{ }\KeywordTok{qnorm}\NormalTok{(}\FloatTok{0.975}\NormalTok{) }\OperatorTok{*}\StringTok{ }\KeywordTok{sd}\NormalTok{(vector) }\OperatorTok{/}\StringTok{ }\KeywordTok{sqrt}\NormalTok{(}\KeywordTok{length}\NormalTok{(vector)), }\DecValTok{4}\NormalTok{)}
  \CommentTok{#low value of the confidence interval}
\NormalTok{  low_val <-}\StringTok{ }\KeywordTok{round}\NormalTok{(}\KeywordTok{mean}\NormalTok{(vector) }\OperatorTok{+}\StringTok{ }\KeywordTok{qnorm}\NormalTok{(}\FloatTok{0.025}\NormalTok{) }\OperatorTok{*}\StringTok{ }\KeywordTok{sd}\NormalTok{(vector) }\OperatorTok{/}\StringTok{ }\KeywordTok{sqrt}\NormalTok{(}\KeywordTok{length}\NormalTok{(vector)), }\DecValTok{4}\NormalTok{)}
  \CommentTok{#ouput statement}
  \KeywordTok{cat}\NormalTok{(}\KeywordTok{paste0}\NormalTok{(}\StringTok{"The two tail 95% confidence interval, of sample size "}\NormalTok{, }\KeywordTok{length}\NormalTok{(vector), }
      \StringTok{", }\CharTok{\textbackslash{}n}\StringTok{is between the values of "}\NormalTok{, low_val, }\StringTok{" and "}\NormalTok{, high_val))}
\NormalTok{\}}
\CommentTok{# calling the function}
\KeywordTok{confidence_int}\NormalTok{(n_}\DecValTok{5}\NormalTok{)}
\end{Highlighting}
\end{Shaded}

\begin{verbatim}
## The two tail 95% confidence interval, of sample size 5, 
## is between the values of -0.7131 and 0.9716
\end{verbatim}

\begin{Shaded}
\begin{Highlighting}[]
\KeywordTok{set.seed}\NormalTok{(}\DecValTok{1}\NormalTok{)}
\NormalTok{n_}\DecValTok{10}\NormalTok{ <-}\StringTok{ }\KeywordTok{rnorm}\NormalTok{(}\DecValTok{10}\NormalTok{)}

\KeywordTok{confidence_int}\NormalTok{(n_}\DecValTok{10}\NormalTok{)}
\end{Highlighting}
\end{Shaded}

\begin{verbatim}
## The two tail 95% confidence interval, of sample size 10, 
## is between the values of -0.3516 and 0.616
\end{verbatim}

\begin{Shaded}
\begin{Highlighting}[]
\KeywordTok{set.seed}\NormalTok{(}\DecValTok{1}\NormalTok{)}
\NormalTok{n_}\DecValTok{30}\NormalTok{ <-}\StringTok{ }\KeywordTok{rnorm}\NormalTok{(}\DecValTok{30}\NormalTok{)}

\KeywordTok{confidence_int}\NormalTok{(n_}\DecValTok{30}\NormalTok{)}
\end{Highlighting}
\end{Shaded}

\begin{verbatim}
## The two tail 95% confidence interval, of sample size 30, 
## is between the values of -0.2482 and 0.4131
\end{verbatim}

\begin{enumerate}
\def\labelenumi{\arabic{enumi})}
\setcounter{enumi}{1}
\tightlist
\item
  Calculate the confidence intervals for the bootstrap samples where n
  in 1000, 5000, 10000, 100000. I'm going to use a sample of 10 random
  normal variables whose mean and standard deviation approaches 0 and 1
  as n approaches infinity respectively.
\end{enumerate}

\begin{Shaded}
\begin{Highlighting}[]
\KeywordTok{set.seed}\NormalTok{(}\DecValTok{1}\NormalTok{)}
\NormalTok{normVec <-}\StringTok{ }\KeywordTok{rnorm}\NormalTok{(}\DecValTok{10}\NormalTok{)}

\NormalTok{results <-}\StringTok{ }\KeywordTok{boot}\NormalTok{(normVec, f , }\DataTypeTok{R =} \DecValTok{1000}\NormalTok{)}
\KeywordTok{boot.ci}\NormalTok{(results)}
\end{Highlighting}
\end{Shaded}

\begin{verbatim}
## BOOTSTRAP CONFIDENCE INTERVAL CALCULATIONS
## Based on 1000 bootstrap replicates
## 
## CALL : 
## boot.ci(boot.out = results)
## 
## Intervals : 
## Level      Normal              Basic         
## 95%   (-0.3180,  0.5941 )   (-0.3419,  0.5719 )  
## 
## Level     Percentile            BCa          
## 95%   (-0.3075,  0.6063 )   (-0.2630,  0.6567 )  
## Calculations and Intervals on Original Scale
\end{verbatim}

\begin{Shaded}
\begin{Highlighting}[]
\NormalTok{results <-}\StringTok{ }\KeywordTok{boot}\NormalTok{(normVec, f, }\DataTypeTok{R =} \DecValTok{5000}\NormalTok{)}
\KeywordTok{boot.ci}\NormalTok{(results)}
\end{Highlighting}
\end{Shaded}

\begin{verbatim}
## BOOTSTRAP CONFIDENCE INTERVAL CALCULATIONS
## Based on 5000 bootstrap replicates
## 
## CALL : 
## boot.ci(boot.out = results)
## 
## Intervals : 
## Level      Normal              Basic         
## 95%   (-0.3247,  0.5932 )   (-0.3339,  0.5860 )  
## 
## Level     Percentile            BCa          
## 95%   (-0.3216,  0.5983 )   (-0.2900,  0.6259 )  
## Calculations and Intervals on Original Scale
\end{verbatim}

\begin{Shaded}
\begin{Highlighting}[]
\NormalTok{results <-}\StringTok{ }\KeywordTok{boot}\NormalTok{(normVec, f, }\DataTypeTok{R =} \DecValTok{10000}\NormalTok{)}
\KeywordTok{boot.ci}\NormalTok{(results)}
\end{Highlighting}
\end{Shaded}

\begin{verbatim}
## BOOTSTRAP CONFIDENCE INTERVAL CALCULATIONS
## Based on 10000 bootstrap replicates
## 
## CALL : 
## boot.ci(boot.out = results)
## 
## Intervals : 
## Level      Normal              Basic         
## 95%   (-0.3277,  0.5915 )   (-0.3376,  0.5825 )  
## 
## Level     Percentile            BCa          
## 95%   (-0.3181,  0.6020 )   (-0.3020,  0.6251 )  
## Calculations and Intervals on Original Scale
\end{verbatim}

\begin{Shaded}
\begin{Highlighting}[]
\NormalTok{results <-}\StringTok{ }\KeywordTok{boot}\NormalTok{(normVec, f, }\DataTypeTok{R =} \DecValTok{100000}\NormalTok{)}
\KeywordTok{boot.ci}\NormalTok{(results)}
\end{Highlighting}
\end{Shaded}

\begin{verbatim}
## BOOTSTRAP CONFIDENCE INTERVAL CALCULATIONS
## Based on 100000 bootstrap replicates
## 
## CALL : 
## boot.ci(boot.out = results)
## 
## Intervals : 
## Level      Normal              Basic         
## 95%   (-0.3249,  0.5903 )   (-0.3328,  0.5767 )  
## 
## Level     Percentile            BCa          
## 95%   (-0.3123,  0.5972 )   (-0.2924,  0.6221 )  
## Calculations and Intervals on Original Scale
\end{verbatim}

\begin{enumerate}
\def\labelenumi{\arabic{enumi})}
\setcounter{enumi}{2}
\item
  Determing the smallest sample size n in N* that would provide the best
  approximation compared to the theory. According to Efron and
  Tibshirane(1986), a minimum threshold of 250 bootstrap samples is
  necessary to be accurate. With that in mind, they also say 1000 is a
  safe bet to land on an approximate estimation. With this previous work
  done(looking at the difference between the Normal Group and the Basic
  Group), I would look back to problem 2 and do 1000 bootstrap samples
  as my barebone minimum value for the number of replications as it
  provides a decent cut point for while balancing distance from the
  normal bootstrap method.
\item
  Discuss and comment on the results.\\
  As can be seen above, our biggest gains come from 1000 bootstrap
  samples up to 5000 and then we hit dimishing returns for 4 decimal
  places. As an investigation, I'd like to see the barest minimum 250
  bootstrap samples as recommended by Efron and Tibshirane(1986).
\end{enumerate}

\begin{Shaded}
\begin{Highlighting}[]
\NormalTok{results <-}\StringTok{ }\KeywordTok{boot}\NormalTok{(normVec, f, }\DataTypeTok{R =} \DecValTok{250}\NormalTok{)}
\KeywordTok{boot.ci}\NormalTok{(results, }\DataTypeTok{type =} \StringTok{"norm"}\NormalTok{)}
\end{Highlighting}
\end{Shaded}

\begin{verbatim}
## BOOTSTRAP CONFIDENCE INTERVAL CALCULATIONS
## Based on 250 bootstrap replicates
## 
## CALL : 
## boot.ci(boot.out = results, type = "norm")
## 
## Intervals : 
## Level      Normal        
## 95%   (-0.3340,  0.5657 )  
## Calculations and Intervals on Original Scale
\end{verbatim}

Surprisingly, the bootstrap group of 250 does come close to
approximating the confidence interval. That being said, as long as
compute stays cheap. I would rather push for more bootstraps in my
confidences intervals. So 1000, would be amount that I lean on going
forward.


\end{document}
